\chapter{General Addenda}

\lstinputlisting[
  basicstyle=\ttfamily\small,language=XML,breaklines=true,breakatwhitespace=false,label={lst:gemini-prompt},caption={Prompt to apply the Gemini 2.5 Flash model to the \gls{dp} task. Gemini models are trained to output coordinates from 0 to 1000, with the origin at the top-left corner of the image. Additionally, they are trained to provide bounding boxes as tuples in the $(y_0,x_0,y_1,x_1)$ format. In order to maximize the accuracy of the detected bounding boxes, \lstinline|box_2d|, the key used in Google's official documentation, is used to denote the bounding box tuples in the output \gls{json}.}
]{figures/code/gemini_prompt.txt}

\begin{table}[htpb]
  \centering
  \caption[ParsingResultType]{Complete list of classifications permitted to be returned by a \gls{dp} implementation. Each implementation provides a mapping from their native output classifications to the standard set defined here. Some ParsingResultTypes may only be returned from a subset of these implementations.}
  \begin{tabular}{ll}
  \toprule
  Classification               & Description                                                               \\
  \midrule

  ROOT                         & The top-level node containing the entire document structure               \\

  \bfseries TEXTS              &
  TITLE                        & The specific main title of the document                                   \\
  PARAGRAPH                    & Standard body text content                                                \\
  SECTION_HEADER               & Section headings or subheaders within the text body                       \\
  FOOTNOTE                     & Explanatory notes usually placed at the bottom of a page/text             \\

  \bfseries LISTS              &
  LIST                         & A container node for a list of items                                      \\
  LIST_ITEM                    & An individual item within a list                                          \\
  REFERENCE_LIST               & A container node for a list of reference items                            \\
  REFERENCE_ITEM               & An individual item within a reference list                                \\

  \bfseries FIGURES AND TABLES &
  CAPTION                      & Descriptive text immediately accompanying a table or figure               \\
  FIGURE                       & Graphical elements, diagrams, or pictures                                 \\
  TABLE                        & A container node for tabular data                                         \\
  DOC_INDEX                    & A tabular node containing the TOC or other document information
  TABLE_ROW                    & A horizontal row within a table                                           \\
  TABLE_CELL                   & An individual cell containing data within a table row                     \\

  \bfseries MISCELLANEOUS      &
  PAGE_FOOTER                  & Repeating page footer (page numbers, copyright, etc.)                     \\
  KEY_VALUE                    & A specific key-value pair                                                 \\
  PAGE_HEADER                  & Repeating header found at the top of pages (e.g., journal name)           \\
  KEY_VALUE_AREA               & A distinct region grouped by key-value pairs (e.g., article info)         \\
  FORM_AREA                    & A region indicating form content (e.g., text-fields)                      \\
  FORMULA                      & A mathematical formula
  WATERMARK                    & A watermark from the publishing organization

  \bfseries FALLBACK           &
  UNKNOWN                      & Used when the parser cannot determine the element type                    \\
  MISSING                      & Used when the parser returns a classification for which no mapping exists \\
  \bottomrule
\end{tabular}


\end{table}\label{tab:parsing-result-type}

\clearpage

\begin{figure}
  \centering
  \caption[OmniDocBench configuration template]{Template of the configuration file for the content extraction evaluation using OmniDocBench.\ \texttt{\{\{OMNI\_DOC\_PATH\}\}} is to be replaced with the path to the OmniDocBench ground truth file.\ \texttt{\{\{DT\_PATH\}\}} is to be replaced with the directory that the \gls{dp} implementation's predictions are saved in.}\label{fig:omni-doc-config}
  \begin{tabular}[t]{c}
  \begin{lstlisting}
  end2end_eval:
    metrics:
        text_block:
            metric:
                - Edit_dist
        display_formula:
            metric:
                - Edit_dist
        table:
            metric:
                - TEDS
                - Edit_dist
        reading_order:
            metric:
                - Edit_dist
    dataset:
        dataset_name: end2end_dataset
        ground_truth:
            data_path: {{OMNI_DOC_PATH}}
        prediction:
            data_path: {{DT_PATH}}
        match_method: quick_match
        filter:
            language: english
            data_source: academic_literature
  \end{lstlisting}
\end{tabular}


\end{figure}