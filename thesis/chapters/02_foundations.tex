% !Tex root = ../main.tex

\chapter{Foundations}\label{chapter:foundations}

\section{Retrieval-Augmented Generation}
Content:
\begin{itemize}
  \item Definition
  \item Components
  \item Use cases
\end{itemize}
\todo{Other usages of chunking than RAG}

\subsection{Tokenization}
Content:
\begin{itemize}
  \item Definition
  \item What is a token?
\end{itemize}

\subsection{Text Embeddings}
Content:
\begin{itemize}
  \item Definition
  \item Types of embeddings
\end{itemize}

\subsection{Semantic Similarity}
Content:
\begin{itemize}
  \item Definition
  \item Role in RAG
\end{itemize}

\section{Visual Source Attribution}
\subsection{Bounding Boxes}
\begin{itemize}
  \item Definition
  \item Formats of Bounding boxes
  \item What do we mean when we say bounding boxes in the thesis? (normalized, ltrb)
\end{itemize}

\subsection{Intersection over Union}
According to the definition from \textcite{object_detection}, the \gls{iou} between two bounding boxes $BB_{a}$ and $BB_{b}$ is defined as described in \autoref{eq:iou}. The \gls{iou} can take on any value between 0 and 1, where a value of 0 means that there is no overlap between the two bounding boxes, and a value of 1 means that the two bounding boxes are identical. In the context of object detection, \gls{iou} is commonly used to evaluate the accuracy of predicted bounding boxes against ground truth bounding boxes \autocite{object_detection}.

\begin{equation}
  IoU(BB_{a}, BB_{b}) = \frac{\text{Area of intersection of } BB_{a} \text{ and } BB_{b}}{\text{Area of union of } BB_{a} \text{ and } BB_{b}}
  \label{eq:iou}
\end{equation}

\section{Vision-Language Models}
\todo{Maybe need to add explanations for transformers?}\\
Content:
\begin{itemize}
  \item Definition
  \item Role in RAG
\end{itemize}

\section{PDF format}

\section{Document Parsing}

Also known as document content extraction, \gls{dp} aims to convert unstructured and semi-structured documents into structured data formats.\cite{parsingunveiled} During this process elements such as headings, tables, and figures are extracted from the document while preserving their structural relationships. \\

While there are many tools available for document parsing, most of them can be categorized into either modular pipeline systems or end-to-end VLM models.

\subsection{Modular Pipeline Systems}
\begin{itemize}
  \item Definition
  \item Common Pipeline Stages (Document Layout Analysis, OCR, element type dependent modules)
  \item Types of Pipelines (with VLM, VLM only for Element Recognition)
  \item Examples (Docling, Unstructured.io, many proprietary solutions)
  \item Problems (Error Propagation)
\end{itemize}

\subsection{End-to-End VLM models}
\todo{Maybe add something about History?}
\begin{itemize}
  \item Definition
  \item Difference between Specialized and general vlms
  \item Examples
  \item Problems
\end{itemize}

\section{Document Chunking}
Content:
\begin{itemize}
  \item Definition
  \item Role in RAG systems
  \item Different Types: Rule based, File dependent
\end{itemize}
