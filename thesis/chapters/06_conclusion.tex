% !Tex root = ../main.tex

\chapter{Conclusion}\label{chapter:conclusion}
This study encompasses an in-depth comparative analysis of current document segmentation configurations and their application for oncology guideline documents. We developed a modular document segmentation pipeline, proposing a unified methodology that addresses the fragmented landscape of available \gls{dp} implementations. Furthermore, motivated by the lack of traceability of current solutions, we introduced a novel approach to the chunking problem that prioritizes traceability and enables visual source attribution for established strategies. In total, we provide integrations for eight different \gls{dp} implementations and four chunking strategies, facilitating a rigorous evaluation across a multitude of possible module combinations.

By critically examining existing benchmarks, we established a set of evaluation metrics and datasets tailored to the requirements of the oncology guideline documents. After reviewing the capabilities of the \gls{dp} approaches and 28 distinct chunking configurations, our findings indicate that, while \gls{vlm}-based approaches demonstrate significant potential for \gls{dp}, especially for content extraction, they remain limited by their output variability and positional inaccuracy. Consequently, we conclude that current pipeline-based approaches, specifically MinerU 2.5 Pipeline and Docling, offer the stability and completeness required for the data preparation of the \gls{rag}-based medical knowledge assistant. Our evaluation of the chunking techniques showed that there is no indication that complex breakpoint-based semantic chunking warrants its increased computational requirements. Instead, we come to the conclusion that recursive character chunking, if configured correctly, is likely to provide the best performance for the oncology guideline domain.

As the field of \gls{dp} is continuously evolving, a primary focus in the development of the document segmentation pipeline was to ensure a modular architecture that supports the integration of emerging techniques. We therefore encourage future research towards exploring novel \gls{dp} approaches and applying our proposed methodology to additional chunking techniques.

Another direction that could be explored in future experiments involves addressing the limitations of our current evaluation framework. In particular, expanding the performance assessment of \gls{dp} implementations on multi-page \gls{pdf} documents would bridge a significant gap in the existing research landscape.